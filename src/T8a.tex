\documentclass[convert]{standalone}
\usepackage[RPvoltages]{circuitikz}
\begin{document}
\begin{circuitikz}
\draw 
(0,0) node[npn](Q1){$Q_1$}
(Q1.E) node[vee]{-5V}
(Q1.B) to[R, l_=$R_{16}$] ++(-2,0) 
to[Do=$D_4$, invert, -*] ++(-2,0) coordinate(A2)
to[C, l_=$C_3$, -*] ++(-2,0) coordinate(A1)
to[push button] ++ (0,2)
node[vcc]{+5V}
(A1) to[R, l_=$R_{13}$] ++(0,-2)
node[vee]{-5V}
(A2) to[R, l_=$R_{14}$] ++(0,-2)
node[vee]{-5V}
(Q1.B) to[R, l_=$R_{17}$, *-] ++(0,-2)
node[vee]{-5V}

(Q1.C) to[R=$R_{18}$] ++(0,2) node[vcc]{+5V}
(Q1.C) to[inline invschmitt, a=U1, *-o] ++(2.5,0)
node[above]{RESET}
-- ++(1,0) coordinate(RESET)
|- ++(0.5,2) node[right]{peak detector reset}
++ (0, -0.5) rectangle ++(3.1,1)
(RESET) to[short, *-] ++(0.5,0) node[right]{TRIGGER}
++ (0, -0.5) rectangle ++(2,1)
(RESET) |- ++(0.5,-2) node[right]{integrador reset}
++ (0, -0.5) rectangle ++(2.6,1)
\end{circuitikz}
\end{document}
