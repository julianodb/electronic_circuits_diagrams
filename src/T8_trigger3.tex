\documentclass[convert]{standalone}
\usepackage[RPvoltages]{circuitikz}
\begin{document}
\begin{circuitikz}
\draw 
(0,0) node[npn](Q1){$Q_7$}
(Q1.C) to[R=$R_{15}$] ++(0,2) node[vcc]{+2.5V}
(Q1.E) node[vee]{-2.5V}
(Q1.B) to[R, l_=$R_{14}$] ++(-2,0) 
to[Do=$D_1$, invert] ++(-2,0) coordinate(A2)
to[C, l_=$C_3$, -o] ++(-2,0) coordinate(A1)
node[above]{$V_{in}$}
(A2) to[R, l_=$R_{12}$, *-] ++(0,-2)
node[vee]{-2.5V}
(Q1.B) to[R, l_=$R_{13}$, *-] ++(0,-2)
node[vee]{-2.5V}
(Q1.E) ++(0.5,0)

(Q1.C) node[left]{$\overline{V_{TRIGGER}}$}
to[R=$R_{39}$, *-] ++(2,0) 
node[pnp, anchor=B](Q2){$Q_{13}$}
(Q2.C) to[short, -o] ++(1,0) node[above]{$V_+$} node[below]{monoestable}
(Q2.E) node[vcc]{+2.5V}
;
\end{circuitikz}
\end{document}
