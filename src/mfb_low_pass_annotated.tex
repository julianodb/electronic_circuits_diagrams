\documentclass[convert]{standalone}
\usepackage[RPvoltages]{circuitikz}
\begin{document}
\begin{circuitikz}
\ctikzset{resistors/scale=0.8, capacitors/scale=0.7}
\draw (0,0) node[ground]{} --++(0,0.5)
node[op amp, anchor=+](OA){OA1}
(OA.-) --++(-0.5,0) coordinate(minus)
to[generic=$R_2$] ++(-2,0) coordinate(vx)
to[generic=$R_1$, -o] ++(-2,0) node[above]{$v_i$}

(vx) to[generic=$\frac{1}{C_1 s}$, *-] ++(0,-2) node[ground]{}
(minus) to[short, *-] ++(0,1)
to[generic=$\frac{1}{C_2 s}$] ++(2.6,0) coordinate(C2)
to[short, -*] (C2 -| OA.out)

(vx) --++(0,2)
to[generic=$R_3$]++(4,0) 
-| (OA.out)

(OA.out) to[short, *-o] ++(1,0) node[above]{$v_o$}
;
\draw[color=blue]
(vx) ++(-0.6,0) to[open,i=$i_1$] ++(0.3,0) 
(vx)  to[open,i=$i_2$] ++(0,-0.3) 
(vx)  to[open,i=$i_3$] ++(0.3,0) 
(vx) ++(0,0.5) to[open,i=$i_4$] ++(0,0.3) 
(minus) ++(0,0.2) to[open,i=$i_3$] ++(0,0.3) 
(vx) ++(0,2) node[above]{$v_x$}
;
\end{circuitikz}
\end{document}