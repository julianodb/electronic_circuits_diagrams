\documentclass[convert]{standalone}
\usepackage[RPvoltages]{circuitikz}
\begin{document}
\begin{circuitikz}\ctikzset{resistors/scale=0.8, capacitors/scale=0.7}
\draw (0,0) -- ++(0.5,0) coordinate(VI) 
to[short, -o] ++(1,0) node[above]{$v_2$}
(VI) node[anchor=south east]{$-$}
to[sV, l=$v_d$, a=generador de funciones, *-] ++(0,2) 
node[anchor=north east]{$+$}
to[short, -o] ++(1,0) node[above]{$v_1$}

;
\ctikzset{resistors/width=1.5, resistors/zigs=5}
\draw (0,0)
--++(-0.5,0)
node[potentiometershape, rotate=-90,  anchor=wiper](POT){} 
(POT.270) node[left]{$R_{pot}$}
(POT.right) --++(0,-0.5) node[ground]{}
;
\ctikzset{resistors/width=1, resistors/zigs=3}
\draw
(POT.left) --++(0,0.5) node[vcc]{$V_{CC}$}
;
\end{circuitikz}
\end{document}