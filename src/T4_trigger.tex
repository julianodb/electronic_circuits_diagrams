\documentclass[convert]{standalone}
\usepackage[RPvoltages]{circuitikz}
\usepackage{cancel}
\begin{document}
\begin{circuitikz}
\draw 
(0,0) node[npn](Q1){$Q_6$}
(Q1.C) to[R=$R_{16}$] ++(0,2) node[vcc]{+5V}
(Q1.E) node[ground]{}
(Q1.B) to[R, l_=$R_{15}$] ++(-2,0) 
to[Do, invert] ++(-2,0) coordinate(A2)
to[C, l_=$C_5$, -*] ++(-2,0) coordinate(A1)
node[op amp, anchor=out](OA){COMP.C}
(OA.out) to[R=$R_{31}$] ++(0,2) node[vcc]{+5V}

(A2) to[R, l2_=$\xcancel{R_{13}}$ and $R_{29}$, *-] ++(0,-2)
node[ground]{}
(Q1.B) to[R, l2_=\xcancel{$R_{14}$} and $R_{30}$, *-] ++(0,-2)
node[ground]{}
(Q1.C) to[short, *-o] ++(0.5,0) node[above]{$V_{out}$}

(OA.-) ++(-1,-0.2) 
node[potentiometershape, rotate=-90,  anchor=wiper](POT){} 
(POT.270) node[left]{$R_{pot2}$}
(POT.left) to[R=$R_{22}$] ++(0,2)
node[vcc]{+5V}
(POT.right) node[ground]{}

++(0,-1.5)
node[left]{$\frac{dV_{TEMP}}{dt}$} 
to[short, o-] ++(0.5,0)
;
\end{circuitikz}
\end{document}
