\documentclass[convert]{standalone}
\usepackage[RPvoltages]{circuitikz}
\begin{document}
\begin{circuitikz}
\draw
(0,0) node[npn](Q4){$Q_4$}
(Q4.E) node[ground]{}
(Q4.C) to[R=$R_{8}$, *-] ++(0,2) node[vcc]{+5V}
(Q4.B) to[R=$R_{7}$, *-] ++(-2,0) coordinate(Q4B)

++(-3,0)

node[npn, xscale=-1, anchor=B](Q3){\ctikzflipx{$Q_3$}}
(Q3.E) node[ground]{}
(Q3.C) to[R=$R_{6}$, *-] ++(0,2) node[vcc]{+5V}
(Q3.B) to[R, l_=$R_{9}$, *-] ++(2,0) coordinate(Q3B)

(Q4B) -- (Q3B |- Q3.C) -- (Q3.C)
(Q3B) -- (Q4B |- Q4.C) -- (Q4.C)

(Q4.C) to[short,-o] ++(1,0) node[above]{$V_{PTT}$}

(Q4.B) --++(0,-1)  node[npn, anchor=C](Q6){$Q_6$}
(Q6.B) to[R, l_=$R_{46}$, -o] ++(0,-2) node[right]{SET}
(Q6.E) node[ground]{}

(Q3.B) --++(0,-1)  node[npn, xscale=-1, anchor=C](Q5){\ctikzflipx{$Q_5$}}
(Q5.B) to[R=$R_{45}$, -o] ++(0,-2) node[left]{RESET}
(Q5.E) node[ground]{}
;
\end{circuitikz}
\end{document}
