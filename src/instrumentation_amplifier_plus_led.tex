\documentclass[convert]{standalone}
\usepackage[RPvoltages]{circuitikz}
\begin{document}
\begin{circuitikz}
\draw
(0,0) node[above]{$V_1$} to[short, o-] ++(1,0)
node[op amp, noinv input up, anchor=+](OA1){}
(OA1.-) -- ++(0,-1) coordinate(RG1)
to[R=$R_G$, *-*] ++(0,-2) coordinate(RG2)
-- ++(0,-1)
node[op amp, anchor=-](OA2){}
(OA2.+) to[short, -o] ++(-1,0) node[above]{$V_2$}
(RG1) to[R=$R_5$] (RG1 -| OA1.out)
-- (OA1.out) to[R=$R_1$, *-*] ++(2,0) coordinate(R2a)
to[R=$R_2$] ++(3,0) coordinate(R2b)
-- (R2b |- RG1)
-- ++(0,-1)
node[op amp, anchor=out](OA3){}
to[short, *-o] ++(1,0) node[above]{$V_o$}
to[R=$R_B$] ++(2,0) node[npn, anchor=B](Q1){}
(Q1.E) node[ground]{}
(Q1.C) to[R=$R_L$] ++(0,2)
to[leDo, invert] ++(0,2) node[vcc]{$V_{CC}$}
(RG2) to[R=$R_6$] (RG2 -| OA2.out)
-- (OA2.out) to[R=$R_3$, *-*] ++(2,0) coordinate(R2c)
to[R=$R_4$] ++(3,0) node[ground]{}
(OA3.-) -| (R2a)
(OA3.+) -| (R2c)
;
\end{circuitikz}
\end{document}
