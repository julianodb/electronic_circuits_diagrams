\documentclass[convert]{standalone}
\usepackage[RPvoltages]{circuitikz}
\begin{document}
\begin{circuitikz}
\ctikzset{bipoles/oscope/waveform=square}
\draw (0,0) node[ground]{} to[sqV, bipole/is voltage=false, v^>=$5 V_{pp}$, *-] (0,4) -- (2,4) coordinate(LEDA) to[leDo, l_=$D_1$] (2,2) to[R, l_=$R_1$] (2,0) -- (0,0);
\draw (3.5,-0.6) node[ground]{} to[R=$R_2$] ++(0,2) to[pDo, l_=$D_2$] ++(0,2) node[vcc](VCC){+12V};
\draw (5.8,0) node[ground]{} to[american voltage source, l=12 V] ++(0,2) node[vcc]{+12V};
\path (2,7); %make Osc. label- appear in figure
\draw (2,6) node[oscopeshape](OSC){Osc.}
(OSC.right) -- ++(0.5,0)  node[ground]{};
\draw[blue] (OSC.in 1) -- ++(0,-0.1) node[anchor=north east, scale=0.5]{CH1} to[short, -*] ( OSC.in 1 |- LEDA);
\draw[red](OSC.in 2) -- ++(0,-0.1) node[anchor=north west, scale=0.5]{CH2} -- ++(0,-0.5) -- ++(2.6,0) -- ++(0,-3.6) coordinate(CH2) to[short, -*] (VCC |- CH2);
\end{circuitikz}
\end{document}
