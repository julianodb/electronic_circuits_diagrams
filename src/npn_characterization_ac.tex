\documentclass[convert]{standalone}
\usepackage[RPvoltages]{circuitikz}
\begin{document}
\begin{circuitikz}
\draw (0,0) node[npn](Q1){$Q_1$}
(Q1.E) node[ground](GND){}
(Q1.C) to[R=$R_C$, -*] ++(0,2) node[vcc](VCC){$V_{CC}$}
(Q1.B) to[R, l_=$R_B$, *-] ++(-2,0) coordinate(VB)
(Q1.B) --++(0,-1.5) coordinate(CC)
to[cC=$C_C$] ++(0,-2) 
to[sV, l=$V_s$] ++(0,-2) node[ground]{}
;
\ctikzset{resistors/width=1.5, resistors/zigs=5}
\draw (VB)
--++(-0.5,0)
node[potentiometershape, rotate=-90,  anchor=wiper](POT){} 
(POT.270) node[left]{$R_{pot}$}
(POT.right) node[ground]{}
(POT.left) |- (VCC)
;
\ctikzset{bipoles/oscope/waveform=sin}
\draw (1,2) node[oscopeshape](OSC){Osc.}
(OSC.right) -- ++(0.5,0)  node[ground]{};
\draw[blue] (OSC.in 1) -- ++(0,-0.1) node[anchor=north east, scale=0.5]{CH1} -- ( OSC.in 1 |- Q1.C) to[short, -*] (Q1.C);
\draw[red](OSC.in 2) -- ++(0,-0.1) node[anchor=north west, scale=0.5]{CH2} -- ++(0,-0.5) coordinate(ch2) -- (ch2 |- CC) to[short, -*] (CC);
\end{circuitikz}
\end{document}