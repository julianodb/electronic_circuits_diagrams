\documentclass[convert]{standalone}
\usepackage[RPvoltages]{circuitikz}
\begin{document}
\begin{circuitikz}
\draw 
(0,0) 
node[npn](Q1){$Q_7$}
(Q1.C) to[R=$R_{15}$] ++(0,2) node[vcc]{+5V}
(Q1.C) to[short, *-o] ++(0.7,0) node[above]{PULSE}
(Q1.E) node[ground]{}
(Q1.B) to[R, l_=$R_{14}$] ++(-2,0) 
to[Do=$D_1$, invert] ++(-2,0) coordinate(A2)
to[C, l_=$C_3$] ++(-2,0) 
-- ++(-0.5,0)
coordinate(A1)

node[op amp, anchor=out, noinv input up](OA){COMP1}
(OA.+) to[short, -o] ++(-1,0) node[above]{PPG}
(OA.-) node[anchor=north east]{$V_{umbral}$}
-- ++(-2,0)
node[potentiometershape, rotate=-90,  anchor=wiper](POT){} 
(POT.right) node[ground]{}
(POT.left) node[vcc]{+5V}

(A1) to[R=$R_{50}$, *-] ++(0,2) node[vcc]{+5V}

(A2) to[R, l_=$R_{12}$, *-] ++(0,-2)
node[ground]{}
(Q1.B) to[R, l_=$R_{13}$, *-] ++(0,-2)
node[ground]{}
(Q1.E) ++(0.5,0)
;
\draw[blue] (-6.5,-3) rectangle (1.5,4)
(-2.5,-3) node[below]{generador de pulso}
;
\end{circuitikz}
\end{document}
