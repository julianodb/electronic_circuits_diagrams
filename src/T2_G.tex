\documentclass[convert]{standalone}
\usepackage[RPvoltages]{circuitikz}
\begin{document}
\begin{circuitikz}
\draw
(0,0) node[above]{$V_A$}
to[R=$4.3\ k\Omega$, o-] ++(2,0)
node[npn, anchor=B](QA){$Q_A$}
(QA.E) node[ground]{}
(QA.C) to[R=$R_{13}$, *-] ++(0,2) node[vcc]{+5V}
(QA.C) --++(0.5,0)
node[npn, anchor=B](Q1){$Q_1$}
(Q1.C) --++(1,0) coordinate(COM)
to[leDo, invert, *-] ++(0,2)
to[R=$R_{12}$] ++(0,2) node[vcc]{+5V}
(Q1.E) node[ground]{}

(COM) --++(1,0)
node[npn, xscale=-1, anchor=C](Q2){\ctikzflipx{$Q_2$}}
(Q2.E) node[ground]{}
(Q2.B) --++(0.5,0)
node[npn, xscale=-1, anchor=C](QB){\ctikzflipx{$Q_B$}}
(QB.E) node[ground]{}
(QB.B) to[R, l=$4.3\ k\Omega$, -o] ++(2,0) node[above]{$V_B$}
(QB.C) to[R=$R_{14}$, *-] ++(0,2) node[vcc]{+5V}
;
\end{circuitikz}
\end{document}
