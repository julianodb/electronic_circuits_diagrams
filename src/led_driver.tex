\documentclass[convert]{standalone}
\usepackage[RPvoltages]{circuitikz}
\begin{document}
\begin{circuitikz}
\draw (0,0)
to[R, l_=$R_1$] ++(0,-2)
to[leDo, l_=$D_1$] ++(0,-2);
\draw (2,1) node[vcc]{5V} -- (2,0)
to[R, l_=$R_2$] ++(0,-2)
to[leDo, l_=$D_2$] ++(0,-2);
\draw (4,0)
to[R, l_=$R_3$] ++(0,-2)
to[leDo, l_=$D_3$] ++(0,-2);
\draw (0,0) to[short, -*] (2,0) to[short, -*] (4,0) -- (4.4,0);
\draw (0,-4) to[short, -*] (2,-4) to[short, -*] (4,-4) -- (4.4,-4);
\path (4,0) -- node {\huge$\dots$} (6,0);
\draw (5.6,0) -- (6,0)
to[R, l_=$R_n$] ++(0,-2)
to[leDo, l_=$D_n$] ++(0,-2)
-- ++(-0.4,0);
\path (6,-4) -- node {\huge$\dots$} (4,-4);
\draw (2,-5) node[npn](Q1){$Q_1$}
(Q1.E) node[ground](GND){}
(Q1.C) -- (2,-4)
(Q1.B) to[R=$R_B$, -o] ++(-2,0) node[above]{$V_i$};
\draw[blue] (-1,-4.3) rectangle ++(8,5)
(7,-4.3) node[anchor=north east]{LED circuit};
\end{circuitikz}
\end{document}