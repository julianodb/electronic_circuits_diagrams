\documentclass[convert]{standalone}
\usepackage[RPvoltages]{circuitikz}
\begin{document}
\begin{circuitikz}
\draw 
(0,0) 
node[op amp, anchor=out](OA){}
(OA.up) --++(0,0.2) node[vcc]{+2.5V}
(OA.down) --++(0,-0.2) node[vee]{-2.5V}
(OA.-) to[short, -o] ++(-1,0) node[above]{$v_i$}
(OA.+) node[anchor=north east]{$V_{umbral}$}
-- ++(-2,0)
node[potentiometershape, rotate=-90,  anchor=wiper](POT){} 
(POT.right) node[ground]{}
(POT.left) node[vcc]{+2.5V}
(POT.270) node[left]{$R_{pot2}$}

(OA.out) to[C, l_=$C_{12}$, -*] ++(2,0) coordinate(VO)
to[R=$R_{13}$] ++(0,2) node[vcc]{+2.5V}
(VO) to[Do=$D_2$, invert] ++(2,0) coordinate(R14)
to[R=$R_{14}$] ++(0,2) node[vcc]{+2.5V}
(R14) to[R=$R_{15}$, *-] ++(2,0)
node[pigfete, anchor=G](Q1){}
(Q1.S) node[vcc]{+2.5V}
(Q1.D) to[generic=$R_{futuro}$] ++(0,-2) node[vee]{-2.5V}
++(0,2) to[short,*-o] ++(0.5,0) node[above]{$V_{o}$}
;
\end{circuitikz}
\end{document}
