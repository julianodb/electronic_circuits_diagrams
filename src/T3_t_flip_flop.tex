\documentclass[convert]{standalone}
\usepackage[RPvoltages]{circuitikz}
\begin{document}
\begin{circuitikz}
\draw
(0,0) node[npn](Q5){$Q_5$}
(Q5.E) node[ground]{}
(Q5.C) to[R, l_=$R_{25}$, *-] ++(0,2) node[vcc]{+3V}
(Q5.B) to[R=$R_{27}$] ++(-2,0) coordinate(Q5B)

++(-3,0)

node[npn, xscale=-1, anchor=B](Q4){\ctikzflipx{$Q_4$}}
(Q4.E) node[ground]{}
(Q4.C) to[R=$R_{24}$, *-] ++(0,2) node[vcc]{+3V}
(Q4.B) to[R, l_=$R_{26}$] ++(2,0) coordinate(Q4B)

(Q5B) -- (Q4B |- Q4.C) -- (Q4.C)
(Q4B) -- (Q5B |- Q5.C) -- (Q5.C)

(Q5.C) ++(-1,0) to[short, *-] ++(0,2.5) coordinate(switch5)
(Q4.C) ++(1,0) to[short, *-] ++(0,2.5) 
to[R=$R_{28}$, -*] ++(2,0)
to[C=$C_{8}$] ++(0,-1.5) node[ground]{}
++(0,1.5)
to[push button, l=ON/OFF] (switch5)

(Q5.C) to[short, -o] ++(1,0) node[above]{$V_{ON}$}
(Q4.C) to[short, -o] ++(-1,0) node[above]{$V_{OFF}$}
;
\end{circuitikz}
\end{document}
